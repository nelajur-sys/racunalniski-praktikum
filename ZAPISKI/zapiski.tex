\documentclass[a4paper, 10pt]{article}

\usepackage[slovene]{babel}
\usepackage[utf8]{inputenc} %vhodno kodiranje
\usepackage[T1]{fontenc} %izhodno kodiranje
\usepackage{amsmath}
\usepackage{amsthm}
\usepackage{hyperref}
\usepackage{graphicx}
\usepackage{booktabs}
\usepackage{xcolor}

\begin{document}

\newcommand{\m}{\mathcal{M}_2}
\newcommand{\pojem}[1]{\emph{\color{purple}#1}}

{\theoremstyle{plain}
\newtheorem{izrek}{Izrek}[section]
\newtheorem{posledica}[izrek]{Posledica}
}

{\theoremstyle{definition}
\newtheorem{definicija}[izrek]{Definicija}
\newtheorem{vaja}[izrek]{Vaja}
}


\newenvironment{magic}[3]{ %novo okolje
   \begin{table}[!ht]
      \centering
      \large
      \caption{#2}
      \label{#3}
      \begin{tabular}{|*{#1}{c|}} \hline
}
{
   \end{tabular}
   \end{table}
}


\title{Izdelava urnikov}
\author{}
\date{}

\maketitle %da se glava izpiše


% POVZETEK
\begin{abstract}
   V članku ...
\end{abstract}

\tableofcontents %KAZALO

\begin{magic}{3}{Začetek magičnega kvadrata}{table:mag3} %uporaba novega okolja
   8 & 1 & 6 \\\hline
   3 & 5 & 7 \\\hline
   4 & 9 & 2 \\\hline
\end{magic}

\begin{table}
   \centering
   \caption{Kvadrat Lo Shu}
   \large
   \label{table:loshu}
   \begin{tabular}{|c|c|c|} \hline
      4 & 9 & 2 \\\hline
      3 & 5 & 7 \\\hline
      8 & 1 & 6 \\\hline
   \end{tabular}
\end{table}

\begin{table}[!ht]
   \centering
   \caption{Število različnih normalnih magičnih kvadratov}
   \label{table:stevila}
   \begin{tabular}{lcccccc} \toprule
         & \multicolumn{5}{c}{točna vrednost} &približek \\ \midrule
         red &1 &2 &3 &4 &5 &6\\
         število kvadratov &1 &0 &1 &880 &275305224 & $(1,7745 \pm 0,0016)10^{19}$ \\ \bottomrule
   \end{tabular}
\end{table}

\section{Program za izdelavo urnika} %razdelek

Spominjam se ...

\subsection{Vhodni podatki} % podrazdelek

Če želimo izdelati urnik...

% ohranja poravnavo, vrstice, odstavke, tako kot je napisano
\begin{verbatim} 
   <profesor>
      <ime>Janez</ime>
      <priimek>Novak</priimek>
      <predmet>
         <imePredmeta>matematika</imePredmeta>
         <steviloUrNaTeden>3</steviloUrNaTeden>
      </predmet>
   </profesor>
\end{verbatim}
% konec okolja verbatim

\noindent
Hierarhija značk...

%NEOŠTEVILČEN SEZNAM
\begin{itemize} 
   \item Razred \emph{Profesor} vsebuje...
   \item Razred \emph{Skupina} predstavlja...
\end{itemize}
% \emph{neki} pomeni LEŽEČE

Naslednji izračun nam poda velikost tabele:
$$\text{velikostTabele} = \text{številoUrNaDan} \cdot \text{številoDelovnihDni}$$
Če bi predpostavili...


\texttt{true} predstavljala %koda na tipkalnem stroju
dosegljivost, \texttt{false} pa zasedenost.


%OŠTEVILČEN SEZNAM
\begin{enumerate}
   
   \item Z zanko se sprehodimo po vektorju vseh predmetov. Predmet na $i$--tem koraku
   označimo s \(\text{predmet}_i\).

   \item Naslednja vgnezdena zanka nas popelje po vseh predavalnicah. Predavalnico na
   $j$--tem koraku označimo s $\text{predavalnica}_j$.

\end{enumerate}

... vmesnik (\underline{G}raphical \underline{U}ser \underline{I}nterface), ki mora urnik... %podčrta

\begin{center} %na sredini
   %\includegraphics{slika.pdf} %vključi sliko
\end{center}

\newpage %nova stran

\begin{itemize}
   \item \url{http://mathworld.wolfram.com/MagicSquare.html} %url link
   \item \url{http://en.wikipedia.org/wiki/Magic_square}
\end{itemize}

\begin{definicija}
   Magični kvadrat reda $n$ je \pojem{normalen}, če v njem nastopajo števila
   \begin{equation}
      \label{eq:numbers}
      % oznaka: eq:numbers
      1, 2, 3, \ldots, n^2-1, n^2.
   \end{equation}
\end{definicija}

...števil (glej \eqref{eq:numbers} na strani \pageref{eq:numbers}) ...
%lahko se sklicujemo na labels

VKLJUČITEV SLIK
% \begin{figure}[!ht]
%    \centering
%    \caption{Dürerjev magični kvadrat}
%    \label{fig:durer}
%    \includegraphics[scale=1.5]{durer.png}
% \end{figure}

\(\cdot\) %central dot
\(\ldots\) %tri pike
\(k_1\) %napisano odspodi
\(\text{predmet}_i\) %besedilo zraven

%VSTAVIMO BIBLIOGRAFIJO
\bibliographystyle{siam}
% \bibliography{ime_datoteke.bib}
% \ref{table:mag3} , \cite{euler}
% \nocite{neki} - VKLJUČENO V BIBLIOGRAFIJI, tudi če ni citirano

\end{document}


% PAKETI:
% amsmath - delaš matematiko, \text{}
% amssymb - logični simboli, \mathbb{R}, \mathbb{N}
% amsthm - izreki (theorem),leme (lemma), dokazi (proof)
% hyperref - \ref, \cite, link
% booktabs - tabele, \toprule, \midrule, \bottomrule
% graphicx - slike
% tikz - diagrami, grafi, skice
% pgfplots - grafi funkcij
% xcolor - barve


% OBLIKOVANJE BESEDILA: 
% \emph - ležeče
% \textbf - krepko
% \textit - italic
% \texttt - pisalni stroj
% \verb
% \url - link
% \noindent - začne čisto levo
% \centering - poravnava v tabeli
% center - okolje poravnano na sredini
% verbatim - ohrani presledke in vrstice


% TABELE:
% table, tabular
% \hline - vodoravna črta
% \toprule, \midrule, \bottomrule - vodoravne črte
% \multicolumn - ena stvar se razteza čez več stolpcev


% dodatne datoteke: \input,






% stolpce columns z okolji column ali ukazi \column.


% simbole: iz tabele matematičnih simbolov,
% posebne prikaze: \frac, \sqrt, \binom, cases, matrike,
% vgrajene funkcije: \sin, \cos, \log, \lim, … (predvsem, da ne napišete samo sin),
% prikazna okolja: equation, align in multline (variante z * in brez),
% pisave: \mathsf, \mathtt, mathrm, \mathbb, \mathcal,
% vstavljanje besedila: \text.
% Pazite, da boste uporabljali ustrezne simbole (npr. \emptyset in ne \phi za prazno množico ali pa \in in ne \epsilon za element).