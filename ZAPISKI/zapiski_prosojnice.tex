\documentclass{beamer}


\usepackage{tikz}
\usetikzlibrary{math}

\usepackage{pgfplots}
\usepgfplotslibrary{external}

\usepackage{array}

\pgfplotsset{compat=1.18}

{\theoremstyle{plain}
\newtheorem{izrek}{Izrek}[section]
\newtheorem{posledica}[izrek]{Posledica}
}

{\theoremstyle{definition}
\newtheorem{definicija}[izrek]{Definicija}
\newtheorem{vaja}[izrek]{Vaja}
}

\newcommand{\Z}{\mathbb{Z}}
\newcommand{\R}{\mathbb{R}}
\newcommand{\C}{\mathbb{C}}

\begin{document}

\title{Matematični izrazi in uporaba paketa \texttt{beamer}}
\subtitle{\emph{Matematičnih} nalog ni treba reševati!}
\institute{Fakulteta za matematiko in fiziko}
\date{}

\begin{frame}
    \titlepage
\end{frame}


%KAZALO VSEBINE
\begin{frame}
    \frametitle{Kratek pregled}
    \tableofcontents%[pausesections]
\end{frame}

% \section{Paket \texttt{beamer}}
% \input{prosojnice/1-paket-beamer.tex}

% \section{Paketa \texttt{amsmath} in \texttt{amsfonts}}
% \input{prosojnice/2-paketa-amsmath-amsfonts.tex}

%  NASLOV PROSOJNICE
\begin{frame}{Posebnosti prosojnic}

	Za prosojnice je značilna uporaba okolja \texttt{frame},
	s katerim definiramo posamezno prosojnico,
	\pause %POSTOPNO ODKRIVANJE
	postopno odkrivanje prosojnic,
	\pause
	ter nekateri drugi ukazi, ki jih najdemo v paketu \texttt{beamer}.
	\pause
	\begin{exampleblock}{Primer} %PRIMER BLOK
		Verjetno ste že opazili, da za naslovno prosojnico niste uporabili
		ukaza \texttt{maketitle}, ampak ukaz \texttt{titlepage}.
	\end{exampleblock}
\end{frame}

\begin{frame}{Poudarjeni bloki}
		\begin{block}{Opomba} %OPOMBA
			Okolja za poudarjene bloke ...
		\end{block}

		\begin{alertblock}{Pozor!} %OPOZORILO
			Začetek poudarjenega bloka...
		\end{alertblock}
\end{frame}

\begin{frame}{Tudi v predstavitvah lahko pišemo izreke in dokaze}
	% Naloga 2.3.2:
	% Oblikujte okolje itemize, tako da se bo njegova vsebina postopoma odkrivala.
	% Ne smete uporabiti ukaza `pause'.
	% Beseda `največje' naj bo poudarjena šele na četrti podprosojnici.

	\begin{izrek}
	   Praštevil je neskončno mnogo.
	\end{izrek}
	\begin{proof}
	   Denimo, da je praštevil končno mnogo.
        % POSTOPNO ODKRIVANJE
	   \begin{itemize}[<+->]
		  \item Naj bo $p$ \alert<4>{največje} praštevilo. 
		  \item Naj bo $q$ produkt števil $1$, $2$, \ldots, $p$.
		  \item Število $q+1$ ni deljivo z nobenim praštevilom, torej je $q+1$ praštevilo.
		  \item To je protislovje, saj je $q+1>p$. \qedhere
	   \end{itemize}
	\end{proof}
 \end{frame}

 \begin{frame}{Matrike}
	
	V pomoč naj vam bo Overleaf dokumentacija o matrikah: %VSE MATRIKE
	https://www.overleaf.com/learn/latex/Matrices

\end{frame}

\begin{frame}{Okolje \texttt{align} in \texttt{align*}}
	% align - OŠTEVILČENO
    % align* - NEOŠTEVILČENO

    % \only - samo na enem mestu
    % \onslide - NA VEČIH MESTIH
	
	\begin{align*}
		(a+b)^n \only{= \ldots }<1>
		\onslide<2-3>{&= (a+b) (a+b) \dots (a+b)} \\
		\onslide<3> {&= a^n + n a^{n-1} b + \dots + \binom{n}{k} a^{n-k} b^k + \dots + n a b^{n-1} + b^n} \\
		&= \sum_{k=0}^n \binom{n}{k} a^{n-k} b^k \\
	\end{align*}
	
\end{frame}


\begin{frame}{Okolje \texttt{multline}}
	% multline - tak douga enačba da je čez več vrstic
	Poišči vse rešitve enačbe
	\begin{multline*}
	(1+x+x^2) \cdot (1+x+x^2+x^3+\ldots+x^9+x^{10}) = \\
	=(1+x+x^2+x^3+x^4+x^5+x^6)^2.
	\end{multline*}
\end{frame}

\begin{frame}{Okolje \texttt{cases}}
	% cases - ZLEPLJENKA
	Dana je funkcija
	\[ f(x,y)=
		\begin{cases}
			\frac{3x^2y-y^3}{x^2+y^2} ; &(x,y)\neq(0,0),\\
			a; &(x,y)=(0,0).
		\end{cases}
	\]
		
	\begin{itemize}
		% displaystyle - PRIKAZNI NAČIN, kao lepše
	\item Določi $a$, tako da izračunaš limito \(\displaystyle\lim_{(x,y)\to(0,0)} f(x). \)
	\item Izračunaj parcialna odvoda $\displaystyle f_x(x,y)$ in $f_y(x,y)$.
	\end{itemize}
\end{frame}

\begin{frame}{Konstrukcija pravokotnice na premico $p$ skozi točko $T$}

	\begin{minipage}{0.55\textwidth} %minipage al neki
		  \begin{itemize}
			 \onslide<1->{\item Dani sta premica $p$ in točka $T$.}
			 \onslide<2->{\item Nariši lok $k$ s središčem v $T$.}
			 \onslide<3->{\item Premico $p$ seče v točkah $A$ in $B$.}
			 \onslide<4->{\item Nariši lok $m$ s središčem v $A$.}
			 \onslide<5->{\item Nariši lok $n$ s središčem v $B$ in z enakim polmerom.}
			 \onslide<6->{\item Loka se sečeta v točki $C$.}
			 \onslide<7->{\item Premica skozi točki $T$ in $C$ je pravokotna na $p$.}
		  \end{itemize}
	\end{minipage}\hfill %da se raztegne levo in desno
	\begin{minipage}{0.45\textwidth}
		\centering
			% \only<1>{\includegraphics[width=50mm]{slike/fig-1.png}}
			% \only<2>{\includegraphics[width=50mm]{slike/fig-2.png}}
			% \only<3>{\includegraphics[width=50mm]{slike/fig-3.png}}
			% \only<4>{\includegraphics[width=50mm]{slike/fig-4.png}}
			% \only<5>{\includegraphics[width=50mm]{slike/fig-5.png}}
			% \only<6>{\includegraphics[width=50mm]{slike/fig-6.png}}
			% \only<7>{\includegraphics[width=50mm]{slike/fig-7.png}}
	\end{minipage}

\end{frame}

\begin{frame}{Odkrivanje tabele po vrsticah}
	Včasih pride prav, da tabelo odkrivamo postopoma po vrsticah.
	\begin{center}
		\begin{tabular}{c|cccc}
		   Oznaka & A & B & C & D \\ \hline
		   \pause
		   X & 1 & 2 & 3 & 4 \\
		   \pause
		   Y & 3 & 4 & 5 & 6 \\
		   \pause
		   Z & 5 & 6 & 7 & 8
		\end{tabular}
	\end{center}
\end{frame}
 

\begin{frame}{Odkrivanje tabele po stolpcih}
	Tabelo lahko odkrivamo tudi po stolpcih, čeprav ni najlažje.

	\begin{center}
		\begin{tabular}{c|>{\onslide<2->}c>{\onslide<3->}c>{\onslide<4->}c>{\onslide<5->}c<{\onslide}}
		   Oznaka & A & B & C & D \\ \hline
		   X & 1 & 2 & 3 & 4 \\
		   Y & 3 & 4 & 5 & 6 \\
		   Z & 5 & 6 & 7 & 8
		\end{tabular}
	\end{center}
\end{frame}


\begin{frame}{Odkrivanje tabele po stolpcih}  %STOLPCI AL NEKI

    \begin{columns}
        \begin{column}{0.5\textwidth}
            LEVI STOLPEC
        \end{column}
        \begin{column}{0.5\textwidth}
            DESNI STOLPEC
        \end{column}
    \end{columns}

\end{frame}



\end{document}